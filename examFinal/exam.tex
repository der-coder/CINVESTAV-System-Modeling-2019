\documentclass[12pt,letterpaper]{article}
\usepackage{graphicx}
\usepackage{ifpdf}

\usepackage{multicol}
\usepackage{tikz}

\usepackage{amssymb}
\usepackage{amsmath}

\usepackage[center]{caption}
\usepackage{subcaption}

\usepackage{hyperref}

\graphicspath{{img/}}

\usepackage{float}
\usepackage{import}
\usepackage{xifthen}
\usepackage{pdfpages}
\usepackage{transparent}
\usepackage{bm}

    \usepackage{listings}
    \lstset{ 
    	language=Matlab,                		% choose the language of the code
    %	basicstyle=10pt,       				% the size of the fonts that are used for the code
    	numbers=left,                  			% where to put the line-numbers
    	numberstyle=\footnotesize,      		% the size of the fonts that are used for the line-numbers
    	stepnumber=1,                   			% the step between two line-numbers. If it's 1 each line will be numbered
    	numbersep=5pt,                  		% how far the line-numbers are from the code
    %	backgroundcolor=\color{white},  	% choose the background color. You must add \usepackage{color}
    	showspaces=false,               		% show spaces adding particular underscores
    	showstringspaces=false,         		% underline spaces within strings
    	showtabs=false,                 			% show tabs within strings adding particular underscores
    %	frame=single,	                			% adds a frame around the code
    %	tabsize=2,                				% sets default tabsize to 2 spaces
    %	captionpos=b,                   			% sets the caption-position to bottom
    	breaklines=true,                			% sets automatic line breaking
    	breakatwhitespace=false,        		% sets if automatic breaks should only happen at whitespace
    	escapeinside={\%*}{*)}          		% if you want to add a comment within your code
    }

\hypersetup{
    colorlinks=true,
    linkcolor=black,
    citecolor = black,
%     filecolor=magenta,      
    urlcolor=black,
%     pdftitle={GSP Toolbox Manual},
%     bookmarks=true
%     pdfpagemode=FullScreen,
}

\usepackage[spanish]{babel}
\decimalpoint

\usepackage{fancyhdr}
 
\pagestyle{fancy}
\fancyhf{}
\rhead{Isaac Ayala Lozano\\194520009 \hspace{2 em}   \textbf{\#2}}
\lhead{2do examen de modelado}
\fancyfoot[R]{\thepage}


\begin{document}
\pagenumbering{7}
% 
% \section{Introducción}
% 
% Se desarrollaron las ecuaciones de movimiento del péndulo esférico mostrado en la figura \ref{fig: pendulum}.
% El valor del potencial de energía $U$ en el estado actual del sistema es cero.
% 
% Las coordenadas generalizadas $\theta$ y $\phi$ describen la posición del sistema. 
% La conversión a coordenadas cartesianas presentadas en el examen escrito fueron:
% 
% \begin{equation}
%  q = \begin{pmatrix}
%       z = b \sin \theta \\
%       y = b \sin \theta \sin \phi\\
%       x = b \sin \theta \cos \phi
%      \end{pmatrix}
% \end{equation}
% 
% Se obtuvieron las siguientes ecuaciones de movimiento mediante la formulación Hamiltoniana.
% 
% \begin{subequations}
%  \begin{align}
%   \dot \theta &= \dfrac{p_\theta}{2 m b^2 \cos^2 \theta}\\
%   \dot p_\theta &= - \dfrac{p_\theta^2 \sin \theta}{2 m b^2 \cos^3 \theta} +  \dfrac{p_\phi^2 \cos \theta}{2 m b^2 \sin^3 \theta} - m g b \sin \theta \\
%   \dot \phi &= \dfrac{p_\theta}{2 m b^2 \sin^2 \theta}\\
%   \dot p_\phi &= 0
%  \end{align}
% \end{subequations}
% 
% 
% 
% \begin{figure}[htb!]
%  \centering 
%  \import{./img/}{pendulum.pdf_tex}
%  \caption{Péndulo esférico.}
%  \label{fig: pendulum}
% \end{figure}
% 
% 
% \section{Simulación}
% 
% Las condiciones de la simulación se presentan en la tabla \ref{table: initial conditions}.
% Se emplean unidades del sistema internacional\footnote{Las unidades del momento provienes de las variables involucradas en su cálculo.}. 
% La constante de gravedad se considera con un valor típico de $9.81 [m/s^2]$.
% 
% % b     = 1; % m
% % m     = 1; % kg
% % g     = 9.81;
% % gamma   = 0.01;   % m
% 
% % tspan = [0 10];
% % b     = 1; % m
% % theta   = pi/10;   % rad/s
% % phi     = pi/5;   % rad/s
% % Ptheta  = 1;   % kg m /s
% % Pphi    = 1;   % kg m/s
% 
% \begin{table}[h]
% \begin{center}
% \centering
% \begin{tabular}{ccccc}
% \hline
% Tiempo de simulación & 10 s & $\gamma$ & 0.01 m &\\
% Masa $m$ & 1 kg & Longitud $b$ & 1 m  & \\
% \hline
% \multicolumn{5}{c}{Condiciones iniciales}\\
%  & $\theta$ & $p_\theta$ & $\phi$ & $p_\phi$\\
% Caso 1 & $\pi/10$ rad & 1 $kg \cdot m^2/s$ & $\pi/5$ rad & 1 $kg \cdot m^2/s$ \\
% \hline
% \end{tabular}
% \end{center}
%  \caption{Condiciones de simulación.}
%  \label{table: initial conditions}
% \end{table}
% 
% \section{Resultados}
% La simulación generó la trayectoria del péndulo para un intervalo de 10 segundos (figura \ref{fig: 3d plot}).
% Se observa que la trayectoria descrita por el sistema no asemeja el movimiento de un péndulo esférico.
% La trayectoria recorrida describe el movimiento cíclico de la masa. 
% Simulaciones con tiempos incrementales no mostraron generar cambio alguno en la trayectoria.
% La masa m está restringida a seguir siempre un ciclo de movimiento como se observa en la figura \ref{fig: time xyz}.\\
% 
% Observando los diagramas de fase de las coordenadas generalizadas (\ref{fig: phase theta phi}) proveen una indicación de qué movimiento está siendo generado.
% La coordenada $\theta$ se limita a un intervalo de valores $\theta \in (0.3,\ 1.1)$ aproximadamente.
% En contraste, la coordenada $\phi$ no está restringida de esa manera.
% Esta segunda variable incrementa sin indicación alguna de detenerse (figura \ref{fig: time theta phi}).
% 
% El crecimiento desmesurado de $\phi$ se debe a que no hay cambio alguno en el momento generalizado $\dot p_\phi =0$.
% Esto puede ser observado en la figura \ref{fig: phase phi p}.
% Contrario a este comportamiento, el diagrama de fase de $\theta - p_\theta$ (figura \ref{fig: phase theta p}) muestra nuevamente la restricción de valores que $\theta$ esperimenta.
% 
% Observando las ecuaciones de movimiento propuestas, se identifican singularidades debido a la presencia de funciones trigonométricas en el denominador de varios términos.
% 
% \begin{figure}[htb!]
%  \centering 
%  \import{./img/}{3Dplot.tex}
%  \caption{Trayectoria recorrida por el péndulo.}
%  \label{fig: 3d plot}
% \end{figure}
% 
% \begin{figure}[htb!]
%  \centering 
%  \import{./img/}{timeXYZ.tex}
%   \caption{Comportamiento de $x, y, z$ respecto al tiempo.}
%  \label{fig: time xyz}
% \end{figure}
% 
% 
% \begin{figure}[htb!]
%  \centering 
%  \import{./img/}{phasePhiPPhi.tex}
%  \caption{Diagrama fase $\phi - p_\phi$.}
%  \label{fig: phase phi p}
% \end{figure}
% 
% 
% 
% \begin{figure}[htb!]
%  \centering 
%  \import{./img/}{phaseThetaPTheta.tex}
%  \caption{Diagrama fase $\theta - p_\theta$.}
%  \label{fig: phase theta p}
% \end{figure}
% 
% \begin{figure}[htb!]
%  \centering 
%  \import{./img/}{timeTRhetaPhi.tex}
%  \caption{Comportamiento de $\theta, \phi$ respecto al tiempo.}
%  \label{fig: time theta phi}
% \end{figure}
% 
% \begin{figure}[htb!]
%  \centering 
%  \import{./img/}{phaseThetaPhi.tex}
%  \caption{Diagrama fase $\theta - p_\phi$.}
%  \label{fig: phase theta phi}
% \end{figure}
% 
% 

\appendix
\section{Código Octave}
\lstinputlisting[language=Matlab]{exam.m}

\end{document}
