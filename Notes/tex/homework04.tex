\section{Homework}

\subsection{Biography: Leonhard Euler}

\begin{center}
 15 April 1707 - 18 September 1783
\end{center}

Born in Basel, Switzerland, he began his education at a young age.
His life would be greatly influenced by Johann Bernoulli, 
a friend of his father, and one of the most distinguished 
mathematicions of the time.
This influence would greatly determine his life, as it was Johann
who tutored him and convinced him to dedicate his life to the study
of mathematics.\\

His contributions to humanity are great in number, 
and span across multiple fields of knowledge. 
Some of his many achievements are listed below.

\begin{itemize}
 \item He laid the groundwork for graph theory by solving the problem of 
Königsberg's bridges. 
\item The Euler equation $\mathrm{e}^{ix} = \cos{x} + i \sin{x}$, 
a contribution so important to mathematics it was called 
`the most remarkable formula in mathematics` by Richard Feynman.
\item Invented calculus of variations, and its most important 
result, the \emph{Euler-Lagrange equation}.
\end{itemize}

\subsection{Biography: Joseph-Louis Lagrange}
\begin{center}
 25 January 1736 - 10 April 1813
\end{center}

Born in Turin, his life had been planned out by his father. 
He studied to become a lawyer at the University of Turin.
He did not show interest in mathematics until the age of seventeen,
an event of utmost importance. He delved into self-study and a year
later he had become an accomplished mathematician. 
He contributed significantly to mathematics and physics, having published
multiple works in France and Germany.

Through his correspondence with Leonhard Euler, they developed 
calculus of variations together. He eventually replaced Euler in 
Frederick of Prussia's court as mathematician. During that time he
published his greatest work \emph{Méchanique analytique}, a work so great
it changed how physical problems were solved by moving from 
Newton's geometrical methods to methods of mathematical analysis.

\subsection{Definitions}
% https://divisbyzero.com/2008/09/22/what-is-the-difference-between-a-theorem-a-lemma-and-a-corollary/
% https://divisbyzero.files.wordpress.com/2008/09/thcorlem.pdf
\begin{itemize}
 \item Theorem. A mathematical statement that is proved using 
 rigorous mathematical reasoning.  
  In a mathematical paper, the term theorem is often reserved for
  the most important results.
 \item Axiom/Postulate. A  statement  that  is  assumed  to  be  true  without  proof.
%  \item Lemma.
 \item Definition. A precise and unambiguous description of the meaning 
 of a mathematical term.  
 It characterizes the meaning of a word by giving all the properties
 and only those properties that must be true.
%  \item Postulate.
 \item Theory.  A set of statements or principles devised to 
 explain a group of facts or phenomena, 
 especially one that has been repeatedly tested or is 
 widely accepted and can be used to make predictions 
 about natural phenomena.
%  https://www.nap.edu/read/11876/chapter/2#11
\end{itemize}


